% Compile with xelatex or pdflatex, XeLaTeX is recommended for Chinese
% texlive 中文 第一个 demo,以前是ctex。
% 20190918 by hua
\documentclass[UTF8]{article}%
\usepackage{ctex,color,graphicx} %中文 ,彩色字体,图

\title{5 mins for latex }
\begin{document}%-开始-----上面的定义和设置,请尽量不要动------------------------
\maketitle          %添加这一句才能够显示标题等信息
注.\textcolor{red}{注意:}\verb|\begin{document}|前面的是设置,尽量不修改。
\verb|\\|=换行。\\
设置步骤:参考,https://blog.csdn.net/wr339988/article/details/66634637,\\
0) texlive2019/TinyTex: 移动硬盘 ,或清华源下载.\textcolor{red}{iso不要解压}.\\
1) WinEdt: 会自动检测latex编译工具链。绿色版TinyTex需手动配置:
Options $\rightarrow$ Execution Modes$\rightarrow$Tex System,
把TeX Root:TeX Live 安装路径,
把TeX Bin:TeX Live的”/bin/win32”,后面的两个选择Auto-detect,
点击应用会自动刷新编译器。熟悉后可配置VSC等更现代化的代码编辑器.\\
2) pdf: 菜单栏:Options$\rightarrow$Execution Modes$\rightarrow$PDF Viewer,
点“Browse”按钮,选择如 "xx\verb|\|SumatraPDF.exe",xx是你机器上的路径。
PDF Viewer窗口下方的“Use--synctex ……”前的选项一定要打勾。每次修改后,按Ctrl+S。\\

公式: $a^2+b^2=c^2$, 及带编号公式:
\begin{equation}
\cos(2 \theta) = \cos ^ 2 \theta - \sin ^ 2 \theta
\end{equation}

注:IEEE,爱思唯尔等都有 latex模板,插图,公式,表格都有模板,复制了修改就ok。或
公式在线转:https://www.latexlive.com/,或bing mathtype转latex.
表格在线转:https://tableconvert.com/ \\

结束,一般期刊投稿都有word模板都,包含CVPR。但是有的定稿要求latex,对同学要求会用就ok。
遇到什么不会的就 bing一下。全部tex代码如下图所示实例:\\
\includegraphics[width=6.5in]{5min4latex-demo}%eps or pdf,1 in = 2.54cm.

\end{document} % -结束---------------------------------------------------